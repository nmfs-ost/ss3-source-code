
 \section{Introduction}\label{sec:intro}
	This manual provides a guide for using the stock assessment program, Stock Synthesis (SS).  The guide contains a description of the input and output files and usage instructions. A technical description of the model itself is in Methot and Wetzel (2013).  SS is programmed using Auto Differentiation Model Builder (ADMB; Fournier 2001.  ADMB is now available at admb-project.org).  SS currently is compiled using ADMB version 11.1 using the Microsoft C++ Optimizing Compiler Version 16.0.  The model and a graphical user interface are available from the NOAA Fisheries Stock Assessment Toolbox website:  http://nft.nefsc.noaa.gov/.  An output processor package, r4ss, in R is available for download from CRAN or GitHub.   Additional information about the package can be located at https://github.com/r4ss/r4ss .
	
\section{New Features 3.30}
		Stock Synthesis version 3.30 has a number of new features:
		\begin{enumerate}
			\item Version conditional read
			\item Forecast allocation group by year
			\item Subseasons
			\item Fleets - flexible ordering
			\item Catch - revised input format
			\item Catch multiplier
			\item Catch - bycatch fleets
		\end{enumerate}
		%\pageref{sec:intro}.
		
\section{File Organization}		
	\subsection{Input Files}
	\begin{enumerate}
		\item starter.ss:   required file containing filenames of the data file and the control file plus other run controls (required).
		\item datafile:  file containing model dimensions and the data with file extension .dat (required)
		\item control file:  file containing set-up for the parameters with file extension .ctl (required)
		\item forecast.ss:  file containing specifications for forecasts (required) 
		\item ss3.par:  previously created parameter file that can be read to overwrite the initial parameter values in the control file (optional)
		\item runnumber.ss:  file containing a single number used as runnumber in output to CumReport.sso and in the processing of profilevalues.ss (optional)
		\item profilevalues.ss:  file contain special conditions for batch file processing (optional)
	\end{enumerate}
	
	\subsection{Output Files}
	\begin{enumerate}
		\item ss3.par, ss3.std, ss33.rep, ss3.cor etc.  standard ADMB output files
		\item echoinput.sso:  This file is produced while reading the input files and includes an annotated echo of the input.  The sole purpose of this output file is debugging input errors.
		\item warning.sso:  This file contains a list of warnings generated during program execution.
		\item checkup.sso:  Contains details of selectivity parameters and resulting vectors.  This is written during the first call of the objective function.
		\item Report.sso:  This file is the primary report file.
		\item CompReport.sso:  Beginning with version 3.03, the composition data has been separated into a dedicated report
		\item Forecast-report.sso:  Output of management quantities and for forecasts
		\item CumReport.sso:  This file contains a brief version of the run output, output is appended to current content of file so results of several runs can be collected together.  This is useful when a batch of runs is being processed.
		\item Covar.sso:  This file replaces the standard ADMB ss3.cor with an output of the parameter and derived quantity correlations in database format
		\item data.ss\textunderscore new:  contains a user-specified number of datafiles, generated through a parametric bootstrap procedure, and written sequentially to this file
		\item control.ss\textunderscore new:  Updated version of the control file with final parameter values replacing the Init parameter values.
		\item starter.ss\textunderscore new:  New version of the starter file with annotations
		\item Forecast.ss\textunderscore new:  New version of the forecast file with annotations.
		\item rebuild.dat:  Output formatted for direct input to Andre Punt's rebuilding analysis package.  Cumulative output is output to REBUILD.SS (useful when doing MCMC or profiles).
	\end{enumerate}

	
	\subsection{Auxiliary Files}
	\begin{enumerate}
		\item SS3-OUTPUT.XLS:   Excel file with macros to read report.sso and display results
		\item SELEX24\textunderscore dbl\textunderscore normal.XLS:
		\begin{enumerate}
			\item This excel file is used to show the shape of a double normal selectivity (option number 20 for age-based and 24 for length-based selectivity) given user-selected parameter values.
			\item Instructions are noted in the XLS file but, to summarize
			\begin{enumerate}
				\item Users should only change entries in a yellow box. 
				\item Parameter values are changed manually or using sliders, depending on the value of cell I5.
			\end{enumerate}
			\item It is recommend that users select plausible starting values for double-normal selectivity options, especially when estimating all 6 parameters
			\item Please note that the XLS does NOT show the impact of setting parameters 5 or 6 to ''-999''.  In SS3, this allows the the value of selectivity at the initial and final age or length to be determined by the shape of the double-normal arising from parameters 1-4, rather than forcing the selectivity at the intial and final age or length to be estimated separately using the value of parameters 5 and 6. 
		\end{enumerate}
		\item SELEX17\textunderscore age\textunderscore randwalk.XLS:
		\begin{enumerate}
			\item This excel file is used to show the shape of age-based selectivity arising from option 17 given user-selected parameter values
			\item Users should only change entries in the yellow box.
			\item The red box is the maximum cumulative value, which is subtracted from all cumulative values.  This is then exponentiated to yield the estimated selectivity curve.  Positive values yield increasing selectivity and negative values yield decreasing selectivity.
		\end{enumerate}
		\item PRIOR-TESTER.XLS:
		\begin{enumerate}
			\item The 'compare' tab of this spreadsheet shows how the various options for defining parameter priors work
		\end{enumerate}
		\item SS-Control\textunderscore Setup.XLS:
		\begin{enumerate}
			\item Shows how to setup an example control file for SS
		\end{enumerate}
		\item SS-Data\textunderscore Input.XLS:
		\begin{enumerate}
			\item Shows how to setup an example data input for SS
		\end{enumerate}
		\item Growth.XLS: 
		\begin{enumerate}
			\item Excel file to test parameterization between the growth curve options within SS.
			\item Instructions are noted in the XLS file but, to summarize
			\begin{enumerate}
				\item Users should only change entries in a yellow box.  
				\item Entries in a red box are used internally, and can be compared with other parameterizations, but should not be changed.
			\end{enumerate}
			\item The SS-VB is identical to the standard VB, but uses a parameterization where length is estimated at pre-defined ages, rather than A=0 and A=Inf.  The Schnute- Richards is identical to the Richards-Maunder, but similarly uses the parameterization with length at pre-defined ages.  The Richards coefficient controls curvature, and if the curvature coefficient = 1, it reverts to the standard VB curve. 
		\end{enumerate}
		\item Movement.XLS:
		\begin{enumerate}
			\item Excel file to explore SS movement parameterization
		\end{enumerate}
	\end{enumerate}
		
\section{Starting SS}
SS runs as a DOS program with text-based input.  The executable is named ss3.exe.  It can be run at the command prompt in a DOS window, or called from another program, such as R or the SS-GUI or a DOS batch file.  See the section in this manual on use of batch file which can allow ss3.exe to reside in a separate directory.  Sometimes you may receive a version of SS with array checking turned on (SS-safe.exe) or without array checking SS\textunderscore opt.exe.  In this case, it is recommended to rename the one you are planning to use to SS3.exe before running it.  Communication with the program is through text files.  When the program first starts, it reads the file STARTER.SS, which must be located in the same directory from which SS is being run.  The file STARTER.SS contains required input information plus references to other required input files, as described in the File Organization section.  Output from SS is as text files containing specific keywords.  Output processing programs, such as the SS GUI, Excel, or R can search for these keywords and parse the specific information located below that keyword in the text file.


\section{Computer Requirements and Recommendations}
SS is compiled to run under DOS with a 32-bit or 64-bit Windows operating system.  It is recommended that the computer have at least a 2.0 Ghz processor and 2 GB of RAM.  In addition SS has now been successfully compiled in Linux.