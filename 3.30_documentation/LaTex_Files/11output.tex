\section{Output Files}
\subsection{Standard ADMB output files}
Standard ADMB files are created by SS.  These are:\\
\\
SS3.PAR – This file has the final parameter values.  They are listed in the order they are declared in SS.  This file can be read back into SS to restart a run with these values (see running SS).\\
\\
SS3.STD – This file has the parameter values and their estimated standard deviation for those parameters that were active during the model run.  It also contains the derived quantities declared as sdreport variables.  All of this information is also report in the covar.sso.  Also, the parameter section of report.sso lists all the parameters with their SS generated names, denotes which were active in the reported run, displays the parameter standard deviations, then displays the derived quantities with their standard deviations.\\
\\
SS3.REP – This report file is created between phases so, unlike report.sso, will be created even if the Hessian fails.  It does not contain as much output as shown in report.sso.\\
\\
SS3.COR – This is the standard ADMB report for parameter and sdreport correlations.  It is in matrix form and challenging to interpret.  This same information is reported in covar.sso.

\subsection{Derived Quantities}
Before listing the derived quantities reported to the sdreport, there are a couple of topics that deserve further explanation.

\subsubsection{Metric for Fishing Mortality}
A generic single metric of annual fishing mortality is difficult to define in a generalized model that admits multiple areas, multiple biological cohorts, dome-shaped selectivity in size and age for each of many fleets.  Several separate indices are provided and others could be calculated by a user from the detailed information in report.sso.

\subsubsection{Equilibrium SPR}
This index focuses on the effect of fishing on the spawning potential of the stock.  It is calculated as the ratio of the equilibrium reproductive output per recruit that would occur with the current year’s F intensities and biology, to the equilibrium reproductive output per recruit that would occur with the current year’s biology and no fishing.  Thus it internalizes all seasonality, movement, weird selectivity patterns, and other factors.  Because this index moves in the opposite direction than F intensity itself, it is usually reported as 1-SPR.  A benefit of this index is that it is a direct measure of common proxies used for F\textsubscript{MSY}, such as F\textsubscript {40\%}.  A shortcoming of this index is that it does not directly demonstrate the fraction of the stock that is caught each year.  The SPR value is also calculated in the benchmarks (see below).  The derived quantities report shows an annual SPR statistic.  The options, as specified in the starter.ss file, are:
\begin{itemize}
	\item 0 = skip
	\item 1 = (1-SPR)/(1-SPR\textsubscript{TGT})
	\item 2 = (1-SPR)/(1-SPR\textsubscript{MSY})
	\item 3 = (1-SPR)/(1-SPR\textsubscript{Btarget})
	\item 4 = raw SPR
\end{itemize}

\subsubsection{F std}
This index provides a direct measure of fishing mortality.  The options are:
\begin{itemize}
	\item 0 = skip
	\item 1 = exploitation(Bio)
	\item 2 = exploitation(Num)
	\item 3 = sum(Frates)
\end{itemize}
The exploitation rates are calculated as the ratio of the total annual catch (in either biomass or numbers as specified) to the summary biomass or summary numbers on Jan 1.  The sum of the F rates is simply the sum of all the apical Fs.  This makes sense if the F method is in terms of instantaneous F (not Pope’s approximation) and if there are not fleets with widely different size/age at peak selectivity, and if there is no seasonality, and especially if there is only one area.  In the derived quantities, there is an annual statistic that is the ratio of the can be annual F\_std value to the corresponding benchmark statistic.  The available options for the denominator are:
\begin{itemize}
	\item 0 = raw
	\item 1 = F/F\textsubscript {SPR}
	\item 2 = F/F\textsubscript {MSY}
	\item 3 = F/F\textsubscript {Btarget}
\end{itemize}

\subsubsection{F-at-Age}
Because the annual F is so difficult to interpret as a sum of individual F components, an indirect calculation of F-at-age is reported at the end of the report.sso file.  This section of the report calculates Z-at-age simply as $ln(N_{a+1,t+1}/N_{a,t})$.  This is done on an annual basis and summed over all areas.  It is done once using the fishing intensities as estimated (to get Z), and once with the F intensities set to 0.0 to get M-at-age.  This latter sequence also provides a measure of dynamic Bzero.  The user can then subtract the table of M-at-age/year from the table of Z-at-age/year to get a table of F-at-age/year.  From this apical F, average F over a range of ages, or other user-desired statistics could be calculated.  Further work within SS with this table of values is anticipated.

\subsubsection{MSY and other Benchmark Items}
The following quantities are included in the sdreport vector mgmt\_quantities, so obtain estimates of variance.  Some additional quantities can be found in the benchmarks section of the forecast\_report.sso.

\begin{center}
	\begin{longtable}{p{4cm} p{11cm}}
		Benchmark Item &  Description\\
		\hline
		\endfirsthead


		Benchmark Item &  Description\\
		\hline
		\endhead
		
		\endfoot
		\hline		
		\endlastfoot
		
		SSB\_Unfished & Unfished reproductive potential (SSB is commonly female mature spawning biomass)\\
		TotBio\_Unfished & Total age 0+ biomass on Jan 1\\
		SmryBio\_Unfished & Biomass for ages at or above the summary age on Jan 1\\
		Recr\_Unfished & Unfished recruitment\\
		SSB\_Btgt & SSB at user specified SSB target\\
		SPR\_Btgt & Spawner potential ratio (SPR) at F intensity that produces user specified SSB target\\
		Fstd\_Btgt & F statistic at F intensity that produces user specified SSB target\\
		TotYield\_Btgt & Total yield at F intensity that produces user specified SSB target\\
		SSB\_SPRtgt & SSB at user specified SPR target (but taking into account the spawner-recruitment relationship)\\
		Fstd\_SPRtgt & F intensity that produces user specified SPR target\\
		TotYield\_SPRtgt & Total yield at F intensity that produces user specified SPR target\\
		SSB\_MSY & SSB at F intensity that is associated with MSY; this F intensity may be directly calculated to produce MSY, or can be mapped to F\_SPR or F\_Btgt\\
		SPR\_MSY & Spawner potential ratio (SPR) at F intensity associated with MSY\\
		Fstd\_MSY & F statistic at F intensity associated with MSY\\
		TotYield\_MSY & Total yield (biomass) at MSY \\
		RetYield\_MSY & Retained yield (biomass) at MSY 
	\end{longtable}
\end{center}

\subsection{Brief cumulative output}
Cum\_Report.sso:  contains a brief version of the run output, which is appended to current content of file so results of several runs can be collected together.  This is especially useful when a batch of runs is being processed.  Unless this file is deleted, it will contain a cumulative record of all runs done in that subdirectory.  The first column contains the run number.  

\subsection{Output for Rebuilder Package}
Output filename is REBUILD.DAT\\
\\
\#Title  \# various run summary outputs\\
SS\#\_default\_rebuild.dat\\
\# Number of sexes\\
2\\
\# Age range to consider (minimum age; maximum age)\\
0 40\\
\# Number of fleets\\
3\\
\# First year of projection (Yinit)\\
2002\\
\# First Year of rebuilding period (Ydecl)\\
1999\\
\# Number of simulations\\
1000\\
\# Maximum number of years\\
500\\
\# Conduct proejctions with multiple starting values (0 = No, 1 = Yes)\\
0\\
\# Number of parameter vectors\\
1000\\
\# Is the maximum age a plus-group (1 = Yes; 2 = No)\\
1\\
\#Generate future recruitments using historical recruitments (1) historical recruits/spawner (2)  or a stock-recruitment (3)\\
3\\
\# Constant fishing mortality (1) or constant Catch (2) projections\\
1\\
\# Fishing mortality based on SPR (1) or actual rate (2)\\
1v
\# Pre-specify the year of recovery (or -1) to ignore\\
-1\\
\# Fecundity-at-age\\
\# 0 1 2 3 4 5 6 7 8 9 10 <deleted values> \\
0 0.000450117 0.00436298 0.0271371 <deleted values> \\
\# Age specific selectivity and weight adjusted for discard and discard mortality\\
\#wt and selex for gender, fleet: 1 1\\
0.146708 0.320119 0.555587 0.830467 <deleted values> \\
0.0122887 0.0351722 0.0838682 0.165479 <deleted values> \\
\#wt and selex for gender ,fleet: 2 1\\
0.150944 0.33768 0.588317 0.874376 <deleted values>\\
0.0127241 0.0380999 0.0922667 <deleted values>\\
\# M and current age-structure in year Yinit: 2002\\
\# gender = 1\\
0.1 0.1 0.1 0.1 0.1 <deleted values>\\
1425.96 797.624 1234.77 428.207 <deleted values>\\
\# gender = 2\\
0.1 0.1 0.1 0.1 0.1 <deleted values>\\
1425.96 797.531 1233.66 <deleted values> \\
\# Age-structure at Ydeclare= 1999\\
598.671 652.739 2925.76 2227.69 <deleted values>\\
598.671 652.666 2923.27 2221.05 <deleted values>\\
\# Year for Tmin Age-structure (set to Ydecl by SS) 1999\\
1999\\
\#  recruitment and biomass\\
\# Number of historical assessment years\\
33\\
\# Historical data\\
\# year recruitment spawner in B0 in R project in R/S project\\
1970 1971 1972 1973 1974 1975 1976 <deleted values> 2001 2002 \\
\#years (with first value representing R0)\\
8853.43  8658.22 8651.96 8645.41 8638.43 8630.75 <deleted values> 1594.53 2075.34 \#recruits; first value is R0 (virgin)\\
63679.5  63679.5 63679.3 63678.3 63673.9 63661.6 <deleted values> 8614.18 7313.2 \#spbio; first value is S0 (virgin)\\
1 0 0 0 0 0 0 0 0 0 0 0 0 0 0 0 0 0 0 0 0 0 0 <deleted values> 0 0  \# in Bzero\\
0 1 1 1 1 1 1 <deleted values> 1 1  0 0 0 \# in R project\\
0 1 1 1 1 1 1 <deleted values> 1 1  0 0 0 \# in R/S project\\
\# Number of years with pre-specified catches\\
0\\
\# catches for years with pre-specified catches go next\\
\# Number of future recruitments to override\\
3\\
\# Process for overriding (-1 for average otherwise index in data list)\\
2000 1 2000\\
2001 1 2001\\
2002 1 2002\\
\# Which probability to product detailed results for (1=0.5; 2=0.6; etc.)\\
3\\
\# Steepness sigma-R Auto-correlation\\
0.610789 0.6 0 \\
\# Target SPR rate (FMSY Proxy); manually change to SPR\_MSY if not using SPR\_target\\
0.5\\
\# Target SPR information: Use (1=Yes) and power\\
0 20\\
\# Discount rate (for cumulative catch)\\
0.1\\
\# Truncate the series when 0.4B0 is reached (1=Yes)\\
0\\
\# Set F to FMSY once 0.4B0 is reached (1=Yes)\\
0\\
\# Maximum possible F for projection (-1 to set to FMSY)\\
-1\\
\# Defintion of recovery (1=now only; 2=now or before)\\
2\\
\# Projection type\\
4\\
\# Definition of the 40-10 rule\\
10 40\\
\# Produce the risk-reward plots (1=Yes)\\
0\\
\# Calculate coefficients of variation (1=Yes)\\
0\\
\# Number of replicates to use\\
10\\
\# Random number seed\\
-99004\\
\# File with multiple parameter vectors \\
rebuild.SS0\\
\# User-specific projection (1=Yes); Output replaced (1->9)\\
0  5  \\
\# Catches and Fs (Year; 1/2/3 (F or C or SPR); value); Final row is -1v
2002 1 1\\
-1 -1 -1\\
\# Split of Fs\\
2002 1\\
-1  1 1 1\\ 
\# Yrs to define TTARGET for projection type 4 (aka 5 pre-specified inputs)\\
2011 2012 2013 2014 2015 2016 2017 2018 \\
\# Time varying weight-at-age (1=Yes;0=No)\\
0\\
\# File with time series of weight-at-age data\\
none\\
\# Use bisection (0) or linear interpolation (1)\\
1\\
\# Target Depletion\\
0.4\\
\# CV of implementation error\\
0\\

\subsection{Bootstrap Data Files}
Data.ss\_new:  contains a user-specified number of data files, generated through a parametric bootstrap procedure, and written sequentially to this file.  These can be parsed into individual data files and re-run with the model.  The first output provides the unaltered input data file (with annotations added).  The second provides the expected values for only the data elements used in the model run.  The third and subsequent outputs provide parametric bootstraps around the expected values.

\subsection{Forecast and Reference Points}
FORECAST-REPORT.sso:  This file contains output of fishery reference points and forecasts.  It is designed to meet the needs of the Pacific Fishery Management Council’s Groundfish Fishery Management Plan, but it should be quite feasible to develop other regionally specific variants of this output.

The vector of forecast recruitment deviations is estimated during an additional model estimation phase.  This vector includes any years after the end of the recrdev time series and before or at the end year.  When this vector starts before the ending year of the time series, then the estimates of these recruitments will be influenced by the data in these final years.  This is problematic, because the original reason for not estimating these recruitments at the end of the time series was the poor signal/noise ratio in the available data.  It is not that these data are worse than data from earlier in the time series, but the low amount of data accumulated for each cohort allows an individual datum to dominate the model’s fit.  Thus, an additional control is provided so that forecast recruitment deviations during these years can receive an extra weighting in order to counter-balance the influence of noisy data at the end of the time series.

An additional control is provided for the fraction of the log-bias adjustment to apply to the forecast recruitments.  Recall that R is the expected mean level of recruitment for a particular year as specified by the spawner-recruitment curve and R’ is the geometric mean recruitment level calculated by discounting R with the log-bias correction factor e-0.5s\^2.  Thus a lognormal distribution of recruitment deviations centered on R’ will produce a mean level of recruitment equal to R.  During the modeled time series, the virgin recruitment level and any recruitments prior to the first year of recruitment deviations are set at the level of R, and the lognormal recruitment deviations are centered on the R’ level.  For the forecast recruitments, the fraction control can be set to 1.0 so that 100\% of the log-bias correction is applied and the forecast recruitment deviations will be based on the R’ level.  This is certainly the configuration to use when the model is in MCMC mode.   Setting the fraction to 0.0 during maximum likelihood forecasts would center the recruitment deviations, which all have a value of 0.0 in ML mode, on R.  Thus would provide a mean forecast that would be more comparable to the mean of the ensemble of forecasts produced in MCMC mode.  Further work on this topic is underway.\\
\\
Note:
\begin{itemize}
	\item Cohorts continue growing according to their specific growth parameters in the forecast period rather than staying static at the endyr values.
	\item Environmental data entered for future years can be used to adjust expected recruitment levels.  However, environmental data will not affect growth or selectivity parameters in the forecast.
\end{itemize}

The top of this file shows the search for F\textsubscript {SPR}  and the search for F\textsubscript {MSY}  so the user can verify convergence.  Note:  if the STD file shows aberrant results, such as all the standard deviations being the same value for all recruitments, then check the F\textsubscript {MSY}  search for convergence. 

The F\textsubscript {MSY} can be calculated, or set equal to one of the other F reference points per the selection made in STARTER.SS.

The reference point output is shown int he table below: 
\begin{center}
	\includegraphics{ManagementReport}
\end{center}


\begin{landscape}
	The forecast is done once using the Target SPR and once using the adjustments specified in the 40:10 section of forecast.ss input.  Each section contains a time series of seasonal biomass and catch, followed by a time series of population numbers-at-age for each morph.
	
	\begin{center}
		\includegraphics{Forecast}
	\end{center}
\end{landscape}

where:
\begin{itemize}
	\item 40:10 is the magnitude of the adjustment of harvest multiplier to implement the OY policy
	\item bio-all is the biomass of all ages
	\item bio-smry is the biomass for ages at or above the summary age
	\item Spawnbio - is the female spawning output
	\item Depletion is the spawnbio divided by the unfished spawnbio
	\item Recruit-0 is the recruitment of age-o fish in this year
	\item Dead\_cat\_B-1 is the total dead (retained plus dead discard) catch in MT for fleet 1
	\item Retain\_B-1 is fleet 1’s retained catch in MT
	\item Equivalent catch in numbers is then reported.
	\item Hrate-1 is the harvest rate, as adjusted by the 40:10 policy.  The units will depend on the F method selected (Pope’s method giving mid-year harvest rate or the continuous F.
	\item Opt=C means that the rate was calculated from an input catch level (and crashed means that this caused an excessive harvest rate.
	\item Opt=R means that the catch was calculated from the target harvest rate.
	\item ABC is equal to the Total-Catch when the 40:10 option is not used (upper portion of table).  When the 40:10 is on (lower table), the ABC is the catch level corresponding to no 40:10 adjustment after accounting for catch in previous year’s from the 40:10.
\end{itemize}

The time series output described above is detailed by season, area, morph and fishery.  It is usually more convenient to have annual values summed across areas, morphs and fisheries.  This is done for the 40:10 output and a subset of these values are replicated in the depletion vector in the sd\_report so that variance estimates can be obtained.  The elements of the depletion vector in the sd\_report are:
\begin{itemize}
	\item depletion level in end year
	\item depletion level in end year+1
	\item MSY (if calculated, else spbio in endyr-1)
	\item BMSY (if calculated, else spbio in endyr)
	\item SPRMSY (if calculated, else spbio in endyr+1) then the time series of:
	\begin{itemize}
		\item Spawning biomass
		\item Recruitment
		\item Depletion level
		\item Total catch (if forecast calculated catch from rates) or sum of fishery-specific harvest rates (if forecast is based on fixed input catch level in this year)
		\item Total exploitation rate (total dead catch divided by the summary biomass at the beginning of the year).
	\end{itemize}
\end{itemize}

\begin{center}
	\includegraphics{HCR}\\
	Two examples of harvest forecast adjustment: one adjusts catch and the other adjusts F.
\end{center}

\subsection{Main Output File, report.sso}
This is the primary output file.  Its major sections are listed below.  

The sections of the output file are:
